\documentclass{article}
 \usepackage[utf8[1] "Transition Matrix:"
               state(t+1)=False state(t+1)=True
state(t)=False              0.8             0.2
state(t)=True               0.9             0.1
[1] "Emission Matrix:"
            emission=False emission=True
state=False            0.7           0.3
state=True             0.5           0.5
[1] "Initial State:"
state(0)=False  state(0)=True 
           0.5            0.5 
[1] "Observations:"
emission(1) emission(2) 
       TRUE        TRUE 
[1] "Step 1: Path Node Probabilities"
[1] 0.135 0.050
[1] "Step 1: Path Step Origins"
[1] "TRUE"  "FALSE"
[1] "Path Node Probabilities"
[1] 0.0324 0.0135
[1] "Step 1: Path Step Origins"
[1] "FALSE" "FALSE"
[1] "Most probable path: TRUE FALSE FALSE"
[1] "Probabality: 0.0324"
[1] "Transition Matrix:"
               state(t+1)=False state(t+1)=True
state(t)=False              0.5             0.5
state(t)=True               0.5             0.5
[1] "Emission Matrix:"
            emission=False emission=True
state=False            0.5           0.5
state=True             0.3           0.7
[1] "Initial State:"
state(0)=False  state(0)=True 
           0.5            0.5 
[1] "Observations:"
emission(1) emission(2) 
      FALSE       FALSE 
[1] "Step 1: Path Node Probabilities"
[1] 0.125 0.075
[1] "Step 1: Path Step Origins"
[1] "FALSE" "FALSE"
[1] "Path Node Probabilities"
[1] 0.03125 0.01875
[1] "Step 1: Path Step Origins"
[1] "FALSE" "FALSE"
[1] "Most probable path: FALSE FALSE FALSE"
[1] "Probabality: 0.03125"
[1] "Transition Matrix:"
               state(t+1)=False state(t+1)=True
state(t)=False              0.2             0.8
state(t)=True               0.2             0.8
[1] "Emission Matrix:"
            emission=False emission=True
state=False            0.1           0.9
state=True             0.2           0.8
[1] "Initial State:"
state(0)=False  state(0)=True 
           0.5            0.5 
[1] "Observations:"
emission(1) emission(2) 
       TRUE       FALSE 
[1] "Forward Values:"
[1] 0.5 0.5
[1] 0.18 0.64
[1] 0.0164 0.1312
[1] "Backward Values:"
[1] 0.1476 0.1476
[1] 0.18 0.18
[1] 1 1
[1] "State Probabilities:"
[1] 0.5 0.5
[1] 0.2195122 0.7804878
[1] 0.1111111 0.8888889
[1] "Transition Matrix:"
               state(t+1)=False state(t+1)=True
state(t)=False              0.6             0.4
state(t)=True               0.7             0.3
[1] "Emission Matrix:"
            emission=False emission=True
state=False            0.3           0.7
state=True             0.7           0.3
[1] "Initial State:"
state(0)=False  state(0)=True 
           0.5            0.5 
[1] "Observations:"
emission(1) emission(2) 
      FALSE        TRUE 
[1] "Forward Values:"
[1] 0.5 0.5
[1] 0.195 0.245
[1] 0.20195 0.04545
[1] "Backward Values:"
[1] 0.2596 0.2352
[1] 0.54 0.58
[1] 1 1
[1] "State Probabilities:"
[1] 0.5246564 0.4753436
[1] 0.4256265 0.5743735
[1] 0.8162894 0.1837106
ne
 Node  & Value \\
\hline
B & FALSE\\
C & FALSE\\
E & FALSE\\
\hline
\end{tabular}
\end{center}
\end{table}
\clearpage
\section{Scheduling}

Provide a complete resource constrained schedule for the actions found in Table~\ref{schActions}. (4 marks)

Please note that the values in the 'After' column refer to the \textit{index} of the action. So '1' refers to Start, not Action 1.

\begin{table}[h!]
\caption{Actions}
\label{schActions}
\begin{center}
\begin{tabular}{ |c|c|c|c|c|c| } 
\hline
 Index & Action & Duration & Uses & Consumes & After \\
\hline
1 & Start & 0 &   & 0 nails & NA\\
2 & Action 1 & 50 &  Saw & 0 nails & 1\\
3 & Action 2 & 45 &  Saw & -1 nail & 1\\
4 & Action 3 & 25 &  Saw & -1 nail & 1\\
5 & Action 4 & 5 &   & 0 nails & 4,3\\
6 & Action 5 & 25 &   & 1 nail & 3,5,4\\
7 & Action 6 & 50 &  Saw & 1 nail & 5,6,2,3,4\\
8 & Action 7 & 10 &  Saw,Hammer & 0 nails & 4\\
9 & Finish & 0 &   & 0 nails & 7,8\\
\hline
\end{tabular}
\end{center}
\end{table}
\clearpage
\section{Multi-Armed Bandit Optimization}

Image we are testing click through rates on three different web layouts. At the current point, the Dirichlet (beta) distributions associated with each layout have the parameters in Table~\ref{MABO1}.\begin{table}[h!]
\caption{Dirichlet (Beta) Parameters for Layout}
\label{MABO1}
\begin{center}
\begin{tabular}{ |c|c|c| } 
\hline
 Layout & Parameter 1 & Parameter 2 \\
\hline
A &  8  &  5 \\
B &  10  &  6 \\
C &  10  &  4 \\
\hline
\end{tabular}
\end{center}
\end{table}

The first value is associated with not clicking through, the second clicking through.

A new person views the site. We generate samples from the distributions to determine which layout is used. These samples are given in Table~\ref{MABO2}.
\begin{table}[h!]
\caption{Samples from Layout Dirichlet (Beta) Distributions}
\label{MABO2}
\begin{center}
\begin{tabular}{ |c|c|c|c|c|c| } 
\hline
 Layout & Sample 1 & Sample 2 & Sample 3 & Sample 4 & Sample 5 \\
\hline
A &  0.56  &  0.41  &  0.25  &  0.36  &  0.41 \\
B &  0.6  &  0.34  &  0.36  &  0.37  &  0.41 \\
C &  0.38  &  0.45  &  0.31  &  0.43  &  0.25 \\
\hline
\end{tabular}
\end{center}
\end{table}


When shown the website with the chosen layout, the person makes a purchase ('clicks through'). Give the new parameters of the three distributions after this event. (2 Marks)
\clearpage
\section{Convolution layers in CNNs}

Tables~\ref{CNN1} to~\ref{CNN3} provide an input matrix and two filter matrices for a convolutional layer in a CNN. Assuming no padding, that stride is [1,1], and that all activation functions are rectifiers, calculate the output of this layer. (2 marks)
\begin{table}[h!]
\caption{Input Matrix}
\label{CNN1}
\begin{center}
\begin{tabular}{ |c|c|c| } 
\hline
1  &  5  &  -1 \\
0  &  -4  &  -1 \\
4  &  -5  &  1 \\
\hline
\end{tabular}
\end{center}
\end{table}
\begin{table}[h!]
\caption{Filter 1}
\label{CNN2}
\begin{center}
\begin{tabular}{ |c|c| } 
\hline
-2  &  3 \\
2  &  4 \\
\hline
\end{tabular}
\end{center}
\end{table}
\begin{table}[h!]
\caption{Filter 1}
\label{CNN3}
\begin{center}
\begin{tabular}{ |c|c| } 
\hline
2  &  3 \\
0  &  3 \\
\hline
\end{tabular}
\end{center}
\end{table}
\clearpage
\section{ Depth-First Search }

Under what conditions could a depth-first search FAIL to find a solution (in a finite search space with at most a single edge between any two nodes)? (1 mark)\clearpage
\section{ Bias-Variance }

Give a basic explanation (as per what was discussed in the course) of the bias and variance components of expected error and their relationship to model complexity. (3 marks)

\end{document}
